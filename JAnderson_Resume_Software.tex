%%%%%%%%%%%%%%%%%%%%%%%%%%%%%%%%%%%%%%%%%
% Developer CV - JA Version
% LaTeX Template
% Version 1.0 (28/1/19)
%
% This template originates from:
% http://www.LaTeXTemplates.com
%
% Authors:
% Jan Vorisek (jan@vorisek.me)
% Based on a template by Jan Küster (info@jankuester.com)
% Modified for LaTeX Templates by Vel (vel@LaTeXTemplates.com) 
% Modified further for personal use by Jen Anderson (andersonn.jen@gmail.com)
%
% License:
% The MIT License (see included LICENSE file)
%
%%%%%%%%%%%%%%%%%%%%%%%%%%%%%%%%%%%%%%%%%

%----------------------------------------------------------------------------------------
%	PACKAGES AND OTHER DOCUMENT CONFIGURATIONS
%----------------------------------------------------------------------------------------

\documentclass[9pt]{developercv} % Default font size, values from 8-12pt are recommended
\usepackage{tikz}
\usetikzlibrary{arrows}
\usepackage{xcolor}
\usepackage{subcaption}
\usepackage{paralist}
\usepackage{pagecolor}

%----------------------------------------------------------------------------------------

\begin{document}
\definecolor{backColor}{HTML}{dfdfdf}

\newpagecolor{backColor}
%----------------------------------------------------------------------------------------
%	TITLE AND CONTACT INFORMATION
%----------------------------------------------------------------------------------------

\begin{minipage}[t]{0.45\textwidth} % 45% of the page width for name
	\vspace{-\baselineskip} % Required for vertically aligning minipages
	
	% If your name is very short, use just one of the lines below
	% If your name is very long, reduce the font size or make the minipage wider and reduce the others proportionately
	\colorbox{black}{{\HUGE\textcolor{white}{\textbf{\MakeUppercase{Jen}}}}} % First name
	
	\colorbox{black}{{\HUGE\textcolor{white}{\textbf{\MakeUppercase{Anderson}}}}} % Last name
	
	\vspace{6pt}
	
	{\huge Software Engineering Manager} % Career or current job title
\end{minipage}
\begin{minipage}[t]{0.3\textwidth} % 27.5% of the page width for the first row of icons
	\vspace{-\baselineskip} % Required for vertically aligning minipages
	
	% The first parameter is the FontAwesome icon name, the second is the box size and the third is the text
	% Other icons can be found by referring to fontawesome.pdf (supplied with the template) and using the word after \fa in the command for the icon you want
	\icon{MapMarker}{12}{Palatine, IL}\\
	\icon{Phone}{12}{+1 708 945 2908}\\
	\icon{At}{12}{\href{mailto:andersonn.jen@gmail.com}{andersonn.jen@gmail.com}}\\	
\end{minipage}
\begin{minipage}[t]{0.275\textwidth} % 27.5% of the page width for the second row of icons
	\vspace{-\baselineskip} % Required for vertically aligning minipages
	
	% The first parameter is the FontAwesome icon name, the second is the box size and the third is the text
	% Other icons can be found by referring to fontawesome.pdf (supplied with the template) and using the word after \fa in the command for the icon you want
	\icon{Globe}{12}{\href{https://andersonnjen.com}{andersonnjen.com}}\\
	\icon{Github}{12}{\href{https://github.com/unybble}{github.com/unybble}}\\
	\icon{Twitter}{12}{\href{https://twitter.com/@unybble}{@unybble}}\\
\end{minipage}

\vspace{0.5cm}

%----------------------------------------------------------------------------------------
%	INTRODUCTION, SKILLS AND TECHNOLOGIES
%----------------------------------------------------------------------------------------

\cvsect{Who Am I?}

\begin{minipage}[t]{0.4\textwidth} % 40% of the page width for the introduction text
	\vspace{-\baselineskip} % Required for vertically aligning minipages
	
Hands-on and encouraging Software Engineering Manager with 10+ years experience coordinating and executing projects, building teams, and creating valuable, cost-effective product. A practiced process-maker, experienced with data analytics and statistical testing, lean code architecture, effective leadership and teamwork. Strongly knowledgeable in the Microsoft stack and open-source FLOSS technologies.
\end{minipage}
\hfill % Whitespace between
\begin{minipage}[t]{0.5\textwidth} % 50% of the page for the skills bar chart
	\vspace{-\baselineskip} % Required for vertically aligning minipages
	\begin{barchart}{5.5}
		\baritem{C\#}{100}
		\baritem{SQL}{70}
		\baritem{Vue.js}{60}
		\baritem{Git/Git Flow}{80}
		\baritem{Analytics}{75}
		\baritem{GoCD/CICD}{60}
	\end{barchart}
\end{minipage}

%\begin{center}
%	\bubbles{5/Eclipse, 6/git, 4/Office, 3/Inkscape, 3/Blender}
%\end{center}

%----------------------------------------------------------------------------------------
% 	WHEEL  ALLOCATION
% Adjusts the size of the wheel:
\def\innerradius{1.2cm}
\def\outerradius{1.6cm}

\newcommand{\wheelchart}[2]{
    % Calculate total
    \pgfmathsetmacro{\totalnum}{0}
    \foreach \value/\colour/\name in {#1} {
        \pgfmathparse{\value+\totalnum}
        \global\let\totalnum=\pgfmathresult
    }

\begin{tikzpicture}

    \node[align=center,text width=2*\innerradius]
    {~{#2}}; % <--- (name is now defined by second     option of the command \wheelchart

  % Calculate the thickness and the middle line of the wheel
      \pgfmathsetmacro{\wheelwidth}{\outerradius-\innerradius}
      \pgfmathsetmacro{\midradius}{(\outerradius+\innerradius)/2}

      % Rotate so we start from the top
      \begin{scope}[rotate=90]

      % Loop through each value set. \cumnum keeps track of where we are in the wheel
      \pgfmathsetmacro{\cumnum}{0}
      \foreach \value/\colour/\name in {#1} {
            \pgfmathsetmacro{\newcumnum}{\cumnum + \value/\totalnum*360}

            % Calculate the percent value
            \pgfmathsetmacro{\percentage}{\value/\totalnum*100}
            % Calculate the mid angle of the colour segments to place the labels
            \pgfmathsetmacro{\midangle}{-(\cumnum+\newcumnum)/2}

            % This is necessary for the labels to align nicely
            \pgfmathparse{
               (-\midangle<180?"west":"east")
            } \edef\textanchor{\pgfmathresult}
            \pgfmathsetmacro\labelshiftdir{1-2*(-\midangle>180)}

            % Draw the color segments. Somehow, the \midrow units got lost, so we add 'pt' at the end. Not nice...
            \fill[\colour] (-\cumnum:\outerradius) arc (-\cumnum:-(\newcumnum):\outerradius) --
            (-\newcumnum:\innerradius) arc (-\newcumnum:-(\cumnum):\innerradius) -- cycle;

            % Draw the data labels
   %         \draw  [*-,thin] node [append after command={(\midangle:\midradius pt) -- (\midangle:\outerradius + 1ex) -- (\tikzlastnode)}] at (\midangle:\outerradius + 1ex) [xshift=\labelshiftdir*0.5cm,inner sep=0pt, outer sep=0pt, ,anchor=\textanchor]{\name: \pgfmathprintnumber{\percentage}\%};


            % Set the old cumulated angle to the new value
            \global\let\cumnum=\newcumnum
        }

      \end{scope}
%      \draw[gray] (0,0) circle (\outerradius) circle (\innerradius);
    \end{tikzpicture}
    }
     \cvsect{Effort Allocation}

 \begin{figure}[!ht]
 
\captionsetup[subfigure]{labelformat=empty}
\begin{subfigure}{.3\textwidth}
\centering
\wheelchart{20/gray/Front-End,  40/white/Back-End,  40/black/SQL}
{\colorbox{lightgray}{
\textcolor{darkgray}{Front-End}}
\break
\colorbox{lightgray}{
\textcolor{white}{Back-End}}
\break
\colorbox{lightgray}{
\textcolor{black}{SQL}}}
\caption{\textbf{STACK}}
\end{subfigure}%
\begin{subfigure}{.3\textwidth}
\centering
\wheelchart{50/gray/Coding, 30/white/Architecture,  20/black/Documentation}
{\colorbox{lightgray}{
\textcolor{darkgray}{Coding}}
\break
\colorbox{lightgray}{
\textcolor{white}{Architecture}}
\break
\colorbox{lightgray}{
\textcolor{black}{Docs}}}
\caption{\textbf{CODE BASE}}
\end{subfigure}%
\begin{subfigure}{.3\textwidth}
\centering
\wheelchart{50/white/Hands-On,  50/black/Collaboration}
{
\colorbox{lightgray}{
\textcolor{white}{Hands-On}}
\break
\colorbox{lightgray}{
\textcolor{black}{Collaboration}}}
\caption{\textbf{DAILY CHARGE}}
\end{subfigure}%%\caption{Two subfigures}
\end{figure}


      
     

%----------------------------------------------------------------------------------------
%	EXPERIENCE
%----------------------------------------------------------------------------------------

\cvsect{Experience}

\begin{entrylist}
	\entry
		{01/2016 - Present}
		{Director of Software Engineering}
		{ECRA Group Inc. | Schaumburg IL}
		{
		 \texttt{MSSQL}\slashsep\texttt{MariaDB}\slashsep\texttt{C\#}\slashsep\texttt{Vue.js}\slashsep\texttt{REST API}\slashsep\texttt{Agile}\slashsep\texttt{Git}\slashsep\texttt{SCRUM}\\
	 \begin{compactitem}
        \item Responsible for leading all software-related projects related to the ECRISS platform through direct supervision of application development.
        \item Directed a group of 6 - 10 developers and professionals in reaching company objectives in a timely manner.
        \item Developing and implementing organized processes for improvement of team efficiency and producing high-quality deliverables. 
        \item Trained all company personnel in novel implementation of AGILE best practices with Kanban and burn-down charts.
        \item   Involved in architecture of new product solutions, code reviews, and organic, seamless upgrades of legacy code. 
        \item Responsible for hiring new team members and conducting training and performance reviews. 
        \item Developed automated tools (SlackBot) within the Agile framework to increase and forecast team productivity.
        \item Developed robust system of near-automatic feature documentation.
        \item Reported status of team to Product, VP of Technology, and CEO on a regular basis.
    \end{compactitem}
    
}
	\entry
		{05/2014 - 12/2015}
		{Software Engineer in Data Science}
		{ECRA Group Inc. | Rosemont IL}
		{
		\texttt{WebServices}\slashsep\texttt{JS}\slashsep\texttt{Python}\slashsep\texttt{CSS}
		 \begin{compactitem}
		 \item Built and managed ECRISS portal, debugged and refactored legacy code. 
		 \item Developed testing protocols and implemented best practice coding standards. 
		 \item Built data analysis tools using basic statistical methodology for clients and associates.
		 \end{compactitem}
		 }
	\entry
		{2011 - present \\\footnotesize{part time}}
		{Lecturer}
		{Purdue University Northwest | Hammond IN}
		{
		\texttt{R-Studio}\slashsep\texttt{SPSS}\slashsep\texttt{Google G-Suite}
		 \begin{compactitem}
		\item Responsible for teaching an online version of statistics 101 covering all of the behavioral methods: descriptives, distribution types, percentile rank with z-score normalization, odds-ratio, and parametric and non-parametric statistical testing. 
		\item Developed the "master" course from scratch, which is currently used to train all incoming nursing students. 
		\item Trained faculty to create and copy content into their individual courses, enforcing a standard format  and structure to deliver to students. 
		 \end{compactitem}}

	\entry
		{10/2019 - 12/2019}
		{Independent Consultant}
		{FTG-Company | Tokyo Japan}
		 {
		 \texttt{Ruby}\slashsep\texttt{AWS S3/Lambda}\slashsep\texttt{CircleCI}\slashsep\texttt{Git}
		 \begin{compactitem}
		 \item Implemented 'Franchise' form within existing SPA Ruby on Rails using Middleman. 
		 \item Updates to lambda function and S3 storage to route incoming form submissions.  	
		 		 \end{compactitem}
}	
	
	\entry
		{2013 - present \\\footnotesize{part time}}
		{Research Assistant}
		{Carthage Vision Lab | Kenosha WI}
		{
		\texttt{Matlab}\slashsep\texttt{Dropbox}\slashsep\texttt{Psychophystics Toolbox}\slashsep\texttt{PsychoPy}
		 \begin{compactitem}
		 \item Wrote and executed psychophysics vision experiments in MatLab Psychophysics Toolbox and PsychoPy3. 
		 \item Maintained documentation for revolving students. 
		 \item Wrote data analysis programs to process large amounts of data generated by experiments. 
		 \item Attended conferences to present findings of experiments. 
		 \end{compactitem}		
		}
\end{entrylist}

%----------------------------------------------------------------------------------------
%	EDUCATION
%----------------------------------------------------------------------------------------

\cvsect{Education}

\begin{entrylist}
	\entry
		{2007 -- 2013}
		{Doctorate of Philosophy}
		{University of Illinois at Chicago}
		{Degree in Behavioral Neuroscience}
	\entry
		{2008 -- 2013}
		{Masters Degree}
		{University of Illinois at Chicago}
		{Degree in Computer Science}
	\entry
		{2007 -- 2010}
		{Masters Degree}
		{University of Illinois at Chicago}
		{Behavioral Neuroscience}
	\entry
		{2002 -- 2007}
		{Bachelors Degree}
		{University of Illinois at Chicago}
		{Psychology (Minor in Computer Science)}
\end{entrylist}

%----------------------------------------------------------------------------------------
%	ADDITIONAL INFORMATION
%----------------------------------------------------------------------------------------

\begin{minipage}[t]{0.3\textwidth}
	\vspace{-\baselineskip} % Required for vertically aligning minipages

	\cvsect{Languages}
	
	\textbf{English} - native\\
	\textbf{Italian} - limited working proficiency
\end{minipage}
\hfill
\begin{minipage}[t]{0.3\textwidth}
	\vspace{-\baselineskip} % Required for vertically aligning minipages
	
	\cvsect{Hobbies}
	
	I love triathlon, lifting weights, and learning. I strive to continuously improve myself as a person.
	\end{minipage}
\hfill
\begin{minipage}[t]{0.3\textwidth}
	\vspace{-\baselineskip} % Required for vertically aligning minipages
	
	\cvsect{Women in STEM}
	
	I am an advocate for encouraging women in STEM and am invited to speak at various events in Chicago.
\end{minipage}

%----------------------------------------------------------------------------------------

\end{document}
